%---------------------------------------------------------------
%       PACKAGES AND OTHER DOCUMENT CONFIGURATIONS
%---------------------------------------------------------------

\documentclass[a4paper]{article}

\usepackage[sc]{mathpazo} % Use the Palatino font
\usepackage[cp1251]{inputenc}
\usepackage[T1]{fontenc} 
\usepackage{cmap}
\linespread{1.05} % Line spacing - Palatino needs more space between lines
\usepackage{microtype} % Slightly tweak font spacing for aesthetics
\usepackage{amssymb} % The amssymb package provides various useful mathematical symbols
\usepackage{amsthm} % The amsthm package provides extended theorem environments
\usepackage{amsmath}
\usepackage[russian, english]{babel}
\usepackage[hmarginratio=1:1,top=32mm,columnsep=20pt]{geometry} % Document margins
\usepackage[hang, small,labelfont=bf,up,textfont=it,up]{caption} % Custom captions under/above floats in tables or figures
\usepackage{booktabs} % Horizontal rules in tables
\usepackage{float} % Required for tables and figures in the multi-column environment - they need to be placed in specific locations with the [H] (e.g. \begin{table}[H])
\usepackage{hyperref} % For hyperlinks in the PDF
\usepackage{lettrine} % The lettrine is the first enlarged letter at the beginning of the text
\usepackage{paralist} % Used for the compactitem environment which makes bullet points with less space between them
\usepackage{abstract} % Allows abstract customization
\renewcommand{\abstractnamefont}{\normalfont\bfseries} % Set the "Abstract" text to bold
\renewcommand{\abstracttextfont}{\normalfont\small\itshape} % Set the abstract itself to small italic text
\usepackage{titlesec} % Allows customization of titles
%\renewcommand\thesection{\Roman{section}} % Roman numerals for the sections
%\renewcommand\thesubsection{\Roman{subsection}} % Roman numerals for subsections
\titleformat{\section}[block]{\large\scshape\centering}{\thesection.}{1em}{} % Change the look of the section titles
\titleformat{\subsection}[block]{\large\centering}{\thesubsection.}{1em}{} % Change the look of the section titles


\usepackage{fancyhdr} % Headers and footers
\pagestyle{fancy} % All pages have headers and footers
\fancyhead{} % Blank out the default header
\fancyfoot{} % Blank out the default footer
\fancyhead[RO]{\small RT\&A, No 4 (40)\\{Volume 11, March \the\year{}}}
\fancyhead[LO]{\small John Smith\\\uppercase{SHORT VERSION OF ARTICLE TITLE}}

% \fancyfoot[CO,LE]{\thepage} % Custom footer text

\usepackage{graphicx}
\usepackage{lipsum}
\usepackage{epstopdf}
\usepackage{comment}
\usepackage{color}
\usepackage{textcase}

\usepackage{geometry}
\geometry{left=3cm}
\geometry{right=3cm}
\geometry{top=3cm}
\geometry{bottom=3cm}

\theoremstyle{definition}
\newtheorem{theorem}{Theorem}
\newtheorem{corollary}{Corollary}
\newtheorem*{remark*}{Remark}
\newtheorem{remark}{Remark}
\newtheorem{lemma}{Lemma}

\renewcommand{\qedsymbol}{$\blacksquare$}
\renewenvironment{proof}{{\bfseries Proof.}}{\qed}

\let\OLDthebibliography\thebibliography
\renewcommand\thebibliography[1]{
  \OLDthebibliography{#1}
  \setlength{\parskip}{0pt}
  \setlength{\itemsep}{0pt plus 0.3ex}
}

%---------------------------------------------------------------
%       TITLE SECTION
%---------------------------------------------------------------

\title{\uppercase\vspace{5mm}\fontsize{16pt}{18pt}\selectfont\textbf{Article title}} % Article title

\author{\large
\textsc{John Smith}\\[2mm] % Your name
$\bullet$\\
\normalsize University of California, USA \\ % Your institution
\normalsize
\normalsize \href{mailto:john@smith.com}{john@smith.com} % Your email address
\vspace{-5mm}
}
\date{}

\vspace{\baselineskip}
\vspace{\baselineskip}

%---------------------------------------------------------------

\begin{document}

\maketitle % Insert title
\thispagestyle{fancy} % All pages have headers and footers

%---------------------------------------------------------------
%       ABSTRACT
%---------------------------------------------------------------

\begin{abstract}
\noindent \lipsum[1] % Dummy abstract text
\end{abstract}
\smallskip
\noindent \textbf{Keywords:} list, keywords, enter, here\\

%---------------------------------------------------------------
%       ARTICLE CONTENTS
%---------------------------------------------------------------

\section{Introduction}

Lorem ipsum dolor sit amet, consectetur adipiscing elit.
\lipsum[2-3] % Dummy text

%------------------------------------------------

\section{Methods}

Maecenas sed ultricies felis \cite{Figueredo:2009dg}. Sed imperdiet dictum arcu a egestas \cite{Joachim:2002}. \\
\begin{compactitem}
\item Donec dolor arcu, rutrum id molestie in, viverra sed diam
\item Curabitur feugiat
\item turpis sed auctor facilisis
\item arcu eros accumsan lorem, at posuere mi diam sit amet tortor
\item Fusce fermentum, mi sit amet euismod rutrum
\item sem lorem molestie diam, iaculis aliquet sapien tortor non nisi
\item Pellentesque bibendum pretium aliquet
\end{compactitem}

\vspace{\baselineskip}

\lipsum[4] % Dummy text

%------------------------------------------------

\section{Results}

\begin{table}[H]
\caption{Example table}
\centering
\begin{tabular}{llr}
\toprule
\multicolumn{2}{c}{Name} \\
\cmidrule(r){1-2}
First name & Last Name & Grade \\
\midrule
John & Doe & $7.5$ \\
Richard & Miles & $2$ \\
\bottomrule
\end{tabular}
\end{table}

\lipsum[5] % Dummy text

\begin{theorem}\label{th1} 
\begin{equation}
\label{eq:emc}
e = mc^2
\end{equation}
\end{theorem}

\begin{proof}
\lipsum[5] 
\end{proof}

\begin{corollary}\label{cor1}
\lipsum[1][1-2] 
\end{corollary}

\begin{figure}[ht]
\leavevmode
\centering \includegraphics[width=0.9\linewidth]{placeholder2.png}
%\centering \includegraphics[width=0.9\linewidth,natwidth=603,natheight=290]{placeholder2.png}
%\includegraphics[width=0.4\linewidth,natwidth=603,natheight=412]{placeholder2.png}
\caption{Figure caption}
\end{figure}

\lipsum[6] % Dummy text

%---------------------------------------------------------------

\section{Discussion}

\subsection{Subsection One}

\lipsum[7] % Dummy text

\subsection{Subsection Two}

\lipsum[8] % Dummy text

%---------------------------------------------------------------
%       REFERENCE LIST
%---------------------------------------------------------------

\begin{thebibliography}{99} % Bibliography - this is intentionally simple in this template

\bibitem{Figueredo:2009dg}
Figueredo, A.~J. and Wolf, P. S.~A. (2009).
\newblock Assortative pairing and life history strategy - a cross-cultural study. \newblock {\em Human Nature}, 20:317--330.

\bibitem{Joachim:2002}
Joachims~J. Learning to Classify Text Using Support Vector Machines: Methods, Theory and Algorithms, Kluwer, 2002.

\end{thebibliography}

%--------------------------------------------------------------

\end{document}